% 
% This is a tex file
%
%\input jnl
%\oneandahalfspace
\documentclass[prd,12pt,superscriptaddress,tightenlines,nofootinbib]{revtex4}
\usepackage{amsmath,amssymb}
\usepackage{bm}
\usepackage{comment}
\usepackage{graphicx}
\usepackage{color}
\usepackage{cancel}
\usepackage{tikz} 
\usetikzlibrary{shapes.misc}
\newcommand*\dstrike[2][thin]{\tikz[baseline] \node [strike out,draw,anchor=text,inner sep=0pt,text=black,#1]{#2};}  
\usepackage{tabularx}
\usepackage{tabulary}

\def\Tr{\rm Tr}
\def\trd{\rm tr_d}
\def\trc{\rm tr_c}
\def\Re{\rm Re}

\begin{document}
    



Let $\vec n$ be a four dimensional vector
$$ \vec n = (n_x, n_y, n_z, n_t) $$
where $n_i $ takes values $0, 1.$ We often equivalently label these 16
vectors by 
$$ n = \sum_{\mu = 1}^4 n_{\mu} 2^{\mu-1}. $$

Let us define 
$$\Gamma(\vec n) = \gamma_1^{n_1}\gamma_2^{n_2}\gamma_3^{n_3}\gamma_4^{n_4}.$$
There are obviously 16 of these $\Gamma$'s and they form a basis for
$ 4 \times 4 $ matrices.

The multiplication table for the $\Gamma$'s is pretty simple to obtain,
$$ \Gamma(\vec n)\Gamma(\vec m) = 
   \eta(\vec n,\vec m)\Gamma(\vec n \oplus \vec m)$$
where $\vec n \oplus \vec m$ is component-wise addition modulo 2 ($ = \vec n 
~.{\bf XOR}.~ \vec m$), and 
$$ \eta(\vec n, \vec m) = \prod_{\mu < \nu}(-)^{n_\nu m_\mu}. $$
Furthermore, with $\eta(\vec n) = \eta(\vec n,\vec n)$,
$$ \Gamma^\dagger(\vec n) = \eta(\vec n) \Gamma(\vec n). $$

Obviously the 16 operators $\overline \psi \Gamma(\vec n)\psi $
exhaust all of the possible mesons.
\smallskip
{\bf Ex.}
$$\vec n = (0, 0, 0, 0) \to 
 \overline \psi \Gamma(\vec n) \psi = \overline \psi \psi$$
$$\vec n = (1, 1, 1, 1) \to 
 \overline \psi \Gamma(\vec n) \psi = \overline \psi \gamma_5 \psi$$
$$\vec n = (0, 0, 1, 0) \to 
 \overline \psi \Gamma(\vec n) \psi = \overline \psi \gamma_3 \psi$$
etc.

A generic $4 \times 4$ matrix $A$ can be decomposed as
$$ A = \sum_{\vec n} a_{\vec n}\Gamma^\dagger(\vec n) $$
where 
$$ a_{\vec n} = {1 \over 4} \Tr \Gamma(\vec n)A. $$

Now consider the correlation function of a meson of type $\vec n$
$$ C_{\vec n} (x) = < \overline \psi(x) \Gamma(\vec n) \psi(x)
 \overline \psi(0) \Gamma^\dagger(\vec n) \psi(0)> $$
$$ ~~~~~~~ =\eta(\vec n) \Tr \left[ \Gamma(\vec n) G(x,0) 
 \Gamma(\vec n) G(0,x) \right]. $$
Here $G(x,0)$ is the quark propagator with $G(0,x) = \gamma_5 
G^\dagger (x,0) \gamma_5$.

Using the decomposition
$$ G(x,0) = \sum_{\vec p} g_{\vec p}(x) \Gamma^\dagger(\vec p)$$
we then find
$$ C_{\vec n} (x) = \eta(\vec n) \sum_{\vec p,\vec q} \trc \left[ 
 g_{\vec p}(x)  g_{\vec q}^\dagger(x) \right] \trd \left[ \Gamma(\vec n) 
 \Gamma^\dagger(\vec p)  \Gamma(\vec n)  \Gamma(\vec {\bf 1}) 
 \Gamma(\vec q) \Gamma(\vec {\bf 1}) \right], $$
where $\vec {\bf 1} = (1,1,1,1)$, {\it i.e.,} $\Gamma(\vec {\bf 1}) =
\gamma_5$. Using the multiplication table for the $\Gamma$'s we find
$$ C_{\vec n} (x) = \eta(\vec n) \sum_{\vec p,\vec q} \eta(\vec p) 
 c_m(\vec n, \vec p, \vec q) \trc \left[ g_{\vec p}(x) g_{\vec q}^\dagger(x) 
 \right] \trd \left[ \Gamma(\vec q\oplus\vec p) \right], $$
where 
$$ c_m(\vec n, \vec p, \vec q) = \eta(\vec n,\vec p) \eta(\vec n,
 \vec {\bf 1}) \eta(\vec q, \vec {\bf 1}) \eta(\vec n \oplus \vec {\bf 1},
 \vec q \oplus \vec {\bf 1}) \eta(\vec n \oplus \vec p, 
 \vec n \oplus \vec q). $$
Since all Dirac matrices, except the identity matrix, are traceless we find
$$ \trd \Gamma(\vec q\oplus \vec p) = 4 \delta_{ \vec p, \vec q}. $$ 

Simple algebra, using the definition for the $\eta$'s reduces the
result to
$$ C_{\vec n} (x) = 4 \eta(\vec n) \sum_{\vec q} \eta(\vec q) 
 \eta(\vec n \oplus \vec q) {\bf SGN}(\vec q) |g_{\vec q}(x)|^2, $$
where
$$ {\bf SGN}(\vec q) = (-)^{q_x + q_y + q_z + q_t}. $$

So in order to do the meson spectroscopy we just have to compute once and 
for all the 16 diagonal $\eta$'s and the ${\bf SGN}$'s. A table is attached
at the end. Note that both $\eta$ and ${\bf SGN}$ are 
$\gamma$-representation independent.

\bigskip

\noindent {\bf B. Vector Currents}
\medskip

We can consider three different vector currents on the lattice, all of
which have the same (naive) continuum limit. On the lattice only one 
current is conserved. Note that we only consider Wilson fermions with the 
Wilson parameter $r$ set to 1. The vector currents are
$$ V^{loc}_{\mu} (x) = \overline \psi(x) \gamma_{\mu} \psi(x), $$
$$ V^{cvc}_{\mu} (x) = {1 \over 2} \left\{ \overline \psi(x) 
 (\gamma_{\mu} - 1) U_{\mu}(x) \psi(x+\mu) + \overline \psi(x+\mu) 
 (\gamma_{\mu} + 1) U^\dagger_{\mu}(x) \psi(x) \right\}, $$
which is the conserved current, and
$$ V^{ncvc}_{\mu} (x) = {1 \over 2} \left\{ \overline \psi(x) 
 \gamma_{\mu} U_{\mu}(x) \psi(x+\mu) + \overline \psi(x+\mu) 
 \gamma_{\mu} U^\dagger_{\mu}(x) \psi(x) \right\}. $$

We want to compute the correlators of these currents with a $\rho$ of
`polarization' $\mu$, where we take $\mu$ to be a space like direction, 
{\it i.e.,} 1, 2 or 3,
$$ C^A_{\mu} (x) = < V^A_{\mu} (x) \overline \psi(0) \gamma_{\mu} 
 \psi(0)>, $$ 
where $A$ is one of $loc$, $cvc$ or $ncvc$. The correlator of the local
current is just the $\gamma_{\mu}$-meson already computed.

We consider the non-conserved current correlator next. We find
$$\eqalign{
 C^{ncvc}_{\mu} (x) = {1 \over 2} \Bigl\{ & ~ \Tr \left[ \gamma_{\mu} 
 U_{\mu}(x) G(x+\mu,0) \gamma_{\mu} \gamma_5 G^\dagger (x,0) \gamma_5
 \right] + \cr
 & ~ \Tr \left[ \gamma_{\mu} U_{\mu}^\dagger(x) G(x,0) \gamma_{\mu} 
 \gamma_5 G^\dagger (x+\mu,0) \gamma_5 \right] \Bigr\}. \cr } $$
Introducing the vector $\vec \mu$ as the unit vector in direction $\mu$ and
going through the same steps as in the computation of the meson propagators
we find
$$\eqalign{
 C^{ncvc}_{\mu} (x) = 2 \sum_{\vec q} & ~ \eta(\vec q) \eta(\vec \mu \oplus 
 \vec q) {\bf SGN}(\vec q) \cr
 & ~ \left\{ \trc \left[ U_{\mu}(x) g_{\vec q}(x+\mu) 
 g_{\vec q}^\dagger (x) \right] + \trc \left[ U_{\mu}^\dagger (x) g_{\vec q}(x) 
 g_{\vec q}^\dagger (x+\mu) \right] \right\}. \cr } $$
The second term is the hermitian conjugate of the first and hence
$$ C^{ncvc}_{\mu} (x) = 4 \sum_{\vec q} \eta(\vec q) \eta(\vec \mu \oplus 
 \vec q) {\bf SGN}(\vec q) \Re \left( \trc \left[ U_{\mu}(x) 
g_{\vec q}(x+\mu)  g_{\vec q}^\dagger (x) \right] \right). $$

We can write the conserved vector current as
$$ V^{cvc}_{\mu} (x) = V^{ncvc}_{\mu} (x) + V^{wils}_{\mu} (x), $$
where
$$ V^{wils}_{\mu} (x) = {1 \over 2} \left\{ - \overline \psi(x) 
 U_{\mu}(x) \psi(x+\mu) + \overline \psi(x+\mu) 
 U^\dagger_{\mu}(x) \psi(x) \right\}. $$
Obviously the correlator has the same decomposition. Thus we need to
consider now only
$$\eqalign{
 C^{wils}_{\mu} (x) = {1 \over 2} \Bigl\{ & ~ \Tr \left[ - U_{\mu}(x) 
 G(x+\mu,0) \gamma_{\mu} \gamma_5 G^\dagger (x,0) \gamma_5
 \right] + \cr
 & ~ \Tr \left[ U_{\mu}^\dagger(x) G(x,0) \gamma_{\mu} 
 \gamma_5 G^\dagger(x+\mu,0) \gamma_5 \right] \Bigr\}. \cr } $$

Again we insert the decomposition of $G(x,0)$ into Dirac matrices. The
spinor part of the correlator is
$$ \trd \left[ \Gamma^\dagger(\vec p) \gamma_{\mu} \Gamma(\vec {\bf 1}) 
 \Gamma(\vec q) \Gamma(\vec {\bf 1}) \right] = 
 4 c_{vw}(\vec \mu, \vec q) \delta_{\vec p, \vec \mu \oplus \vec q}. $$
Here we have defined
$$ c_{vw}(\vec \mu, \vec q) = \eta(\vec \mu \oplus \vec q)
 \eta(\vec \mu, \vec {\bf 1}) \eta(\vec q, \vec {\bf 1}) 
 \eta(\vec \mu \oplus \vec {\bf 1}, \vec q \oplus \vec {\bf 1})
 \eta(\vec \mu \oplus \vec q) 
 = {\bf SGN}(\vec q) \eta(\vec \mu, \vec q) $$
The second equation follows after some algebra.
%% $$ \eta(\vec p) \eta(\vec q) \eta(\vec p \oplus \vec q) =
%% \eta(\vec p, \vec q) \eta(\vec q, \vec p) $$
%% and $\eta(\vec \mu) = 1$.
Then we obtain 
$$\eqalign{
 C^{wils}_{\mu} (x) = 2 \sum_{\vec q} c_{vw}(\vec \mu, \vec q) 
 \Bigl\{ - & ~ \trc \left[ U_{\mu}(x) g_{\vec \mu \oplus \vec q}(x+\mu) 
 g_{\vec q}^\dagger (x) \right] \cr
 + & ~ \trc \left[ U_{\mu}^\dagger (x) g_{\vec \mu \oplus \vec q}(x)
 g_{\vec q}^\dagger (x+\mu) \right] \Bigr\}. \cr } $$

Some algebra shows that
$$ c_{vw}(\vec \mu, \vec \mu \oplus \vec q) = - 
 c_{vw}(\vec \mu, \vec q). $$
Thus, changing the summation variable in the second term of 
$C^{wils}_{\mu} (x)$ from $\vec q$ to $\vec \mu \oplus \vec q$ shows that
it is just the hermitian conjugate of the first term. So we find 
$$ C^{wils}_{\mu} (x) = -  4 \sum_{\vec q} c_{vw}(\vec \mu, \vec q) 
 \Re \left( \trc \left[ U_{\mu}(x) g_{\vec \mu \oplus \vec q}(x+\mu) 
 g_{\vec q}^\dagger (x) \right] \right). $$

The $\eta(\vec n, \vec m)$'s necessary for the computation of the `Wilson
part' of the conserved current correlators have been generated by a 
small Fortran program.

\bigskip

\noindent {\bf C. Axial Currents}
\medskip

There is no (partially) conserved axial current on the lattice. So we just
have to do with lattice axial currents that have the correct naive
continuum limit. We consider two such currents
$$ A^{loc}_{\mu} (x) = \overline \psi(x) \gamma_{\mu} \gamma_5 \psi(x), $$
and
$$ A^{nloc}_{\mu} (x) = {1 \over 2} \left\{ \overline \psi(x) 
 \gamma_{\mu} \gamma_5 U_{\mu}(x) \psi(x+\mu) + \overline \psi(x+\mu) 
 \gamma_{\mu} \gamma_5 U^\dagger_{\mu}(x) \psi(x) \right\}. $$

We want to compute the correlators of the $4^{\rm th}$ component of these 
currents with a pion 
$$ C^B_4 (x) = < A^B_4 (x) \overline \psi(0) \gamma_5 \psi(0)>, $$ 
where $B$ stands for $loc$ or $nloc$. 

We consider the non-local current correlator first. We find
$$\eqalign{
 C^{nloc}_4(x) = {1 \over 2} \Bigl\{ & ~ \Tr \left[ \gamma_4 \gamma_5 
 U_4(x) G(x+\hat 4,0) \gamma_5 \gamma_5 G^\dagger (x,0) 
 \gamma_5 \right] + \cr
 & ~ \Tr \left[ \gamma_4 \gamma_5 U_4^\dagger(x) G(x,0) \gamma_5
 \gamma_5 G^\dagger (x+\hat 4,0) \gamma_5 \right] \Bigr\}. \cr } $$
Once more we insert the decomposition of $G(x,0)$ into Dirac matrices. 
This time the spinor part becomes
$$ \trd \left[ \Gamma(\vec 4) \Gamma(\vec {\bf 1}) \Gamma^\dagger(\vec p) 
 \Gamma(\vec q) \Gamma(\vec {\bf 1}) \right] = 
 4 c_a(\vec 4, \vec q) \delta_{\vec p, \vec 4 \oplus \vec q}. $$
Here we have introduced
$$ c_a(\vec 4, \vec q) = \eta(\vec 4 \oplus \vec q)
 \eta(\vec 4, \vec {\bf 1})  \eta(\vec q, \vec {\bf 1}) 
 \eta(\vec q \oplus \vec {\bf 1}, \vec 4 \oplus \vec {\bf 1})
 \eta(\vec 4 \oplus \vec q) 
 = {\bf SGN}(\vec 4) \eta(\vec q, \vec 4) $$
Then the correlator becomes
$$ C^{nloc}_4(x) = 2 \sum_{\vec q} c_a(\vec 4, \vec q) 
 \left\{ \trc \left[ U_4(x) g_{\vec 4 \oplus \vec q}(x+\hat 4) 
 g_{\vec q}^\dagger (x) \right] + \trc \left[ U_4^\dagger (x) 
 g_{\vec 4 \oplus \vec q}(x) g_{\vec q}^\dagger (x+\hat 4) \right] 
 \right\}. $$

$c_a$ satisfies
$$ c_a(\vec 4, \vec 4 \oplus \vec q) = c_a(\vec 4, \vec q) $$
and the second term above becomes, after changing the summation variable 
from $\vec q$ to $\vec 4 \oplus \vec q$, the hermitian conjugate of the 
first term. Thus we get 
$$ C^{nloc}_4(x) = 4 \sum_{\vec q} c_a(\vec 4, \vec q) 
 \Re \left( \trc \left[ U_4(x) g_{\vec 4 \oplus \vec q}(x+\hat 4) 
 g_{\vec q}^\dagger (x) \right] \right). $$

Analogously we find for the local axial current correlator
$$ C^{loc}_4(x) = 4 \sum_{\vec q} c_a(\vec 4, \vec q) 
 \Re \left( \trc \left[ g_{\vec 4 \oplus \vec q}(x) 
 g_{\vec q}^\dagger (x) \right] \right). $$


\bigskip

\noindent {\bf D. Improved Currents}
\medskip

The local currents considered previously can be ${\cal O}(a)$-improved by
adding
$$ \delta A_\mu(x) = \tilde \partial_\mu 
 \overline \psi(x) \gamma_5 \psi(x), $$
and
$$ \delta V_\mu(x) = \tilde \partial_\nu 
 \overline \psi(x) \gamma_\nu \gamma_\mu \psi(x) $$
respectively with appropriately chosen coefficients. Note that in the
last equation we used $i \sigma_{\mu \nu} = \gamma_\nu \gamma_\mu$. Further
we have defined
$$ \tilde \partial_\mu f(x) = {1 \over 2} \left[ f(x+\mu) - f(x-\mu)
 \right] .$$

We note that the correlator of $\delta A_4(x)$ with $\overline \psi(0)
\gamma_5 \psi(0)$ is, apart from the $\tilde \delta_4$, just the pion
correlator, pion$_1$.

For the vector current we need the correlator of $\delta V_k(x)$ with the
rho, $\overline \psi(0) \gamma_k \psi(0)$. Since we will project on
zero momentum by summing over all $\vec x$, only the $\nu = 4$ part in
$\delta V_k(x)$ will give a non-vanishing contribution
$$ C^{\delta V}_k(x) = < \delta V_k (x) \overline \psi(0) \gamma_k \psi(0)>
 = \tilde \partial_4 \Tr \left[ \gamma_4 \gamma_k G(x,0) \gamma_k
 \gamma_5 G^\dagger (x,0) \gamma_5 \right] .$$ 
The spinor part of the correlator is
$$ \trd \left[ \gamma_4 \gamma_k \Gamma^\dagger(\vec p) \gamma_k
 \Gamma(\vec {\bf 1}) \Gamma(\vec q) \Gamma(\vec {\bf 1}) \right] = 
 4 c_{\delta v}(\vec k, \vec q) \delta_{\vec p, \vec 4 \oplus \vec q}. $$
Here we have defined
$$ c_{\delta v}(\vec k, \vec q) = \eta(\vec 4 \oplus \vec q)
 \eta(\vec 4, \vec k) \eta(\vec k, \vec {\bf 1}) \eta(\vec q, \vec {\bf 1}) 
 \eta(\vec 4 \oplus \vec k, \vec 4 \oplus \vec q)
 \eta(\vec k \oplus \vec {\bf 1}, \vec q \oplus \vec {\bf 1})
 \eta(\vec k \oplus \vec q) 
 = - {\bf SGN}(\vec q) \eta(\vec k \oplus \vec q) $$
One checks easily that
$$ c_{\delta v}(\vec k, \vec 4 \oplus \vec q) =
 c_{\delta v}(\vec k, \vec q) $$
and hence we obtain
$$ C^{\delta V}_k(x) = 4 \sum_{\vec q} c_{\delta v}(\vec k, \vec q)
 \tilde \partial_4 \Re \left( \trc \left[ g_{\vec 4 \oplus \vec q}(x) 
 g_{\vec q}^\dagger (x) \right] \right). $$


%%$$ c_{vw}(\vec \mu, \vec q) = \eta(\vec \mu \oplus \vec q)
%% \eta(\vec \mu, \vec {\bf 1})  \eta(\vec q, \vec {\bf 1}) 
%% \eta(\vec \mu \oplus \vec {\bf 1}, \vec q \oplus \vec {\bf 1})
%% \eta(\vec \mu \oplus \vec q) =
%% {\bf SGN}(\vec q) \eta(\vec \mu, \vec q) $$
%%and
%%$$ c_a(\vec 4, \vec q) = \eta(\vec 4 \oplus \vec q)
%% \eta(\vec 4, \vec {\bf 1}) \eta(\vec q, \vec {\bf 1})
%% \eta(\vec q \oplus \vec {\bf 1}, \vec 4 \oplus \vec {\bf 1})
%% \eta(\vec 4 \oplus \vec q) = 
%% {\bf SGN}(\vec 4) \eta(\vec q, \vec 4) . $$


\eject
\centerline{Table}
\bigskip

\centerline{
\vbox{\offinterlineskip
\hrule\def\tablerule{\noalign{\hrule}}
\def\mystrut{\vrule height 14pt width 0pt depth 5pt}
\def\plus{\hphantom{-}1}\def\minus{-1}
\halign{&\vrule#&\mystrut\quad\hfil#\hfil\quad\cr
& $~n$ && $(n_x,n_y,n_z,n_t)$ && $\eta(n)$ && ${\bf SGN}(n)$ &\cr\tablerule
& ~0 && (0,0,0,0) && \plus  && \plus &\cr\tablerule
& ~1 && (1,0,0,0) && \plus  && \minus&\cr\tablerule
& ~2 && (0,1,0,0) && \plus  && \minus&\cr\tablerule
& ~3 && (1,1,0,0) && \minus && \plus &\cr\tablerule
& ~4 && (0,0,1,0) && \plus  && \minus&\cr\tablerule
& ~5 && (1,0,1,0) && \minus && \plus &\cr\tablerule
& ~6 && (0,1,1,0) && \minus && \plus &\cr\tablerule
& ~7 && (1,1,1,0) && \minus && \minus&\cr\tablerule
& ~8 && (0,0,0,1) && \plus  && \minus&\cr\tablerule
& ~9 && (1,0,0,1) && \minus && \plus &\cr\tablerule
& 10 && (0,1,0,1) && \minus && \plus &\cr\tablerule
& 11 && (1,1,0,1) && \minus && \minus&\cr\tablerule
& 12 && (0,0,1,1) && \minus && \plus &\cr\tablerule
& 13 && (1,0,1,1) && \minus && \minus&\cr\tablerule
& 14 && (0,1,1,1) && \minus && \minus&\cr\tablerule
& 15 && (1,1,1,1) && \plus  && \plus &\cr\tablerule}}
}
\end{document}


